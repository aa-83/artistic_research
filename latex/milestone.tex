\documentclass[12pt, letterpaper]{article}
\usepackage[utf8]{inputenc}
\usepackage[english]{babel}
\usepackage[T1]{fontenc}

\usepackage[
backend=biber,
style=authoryear-icomp
]{biblatex}
\addbibresource{~/documents/latex/bib/bibliografi.bib}
\usepackage{csquotes}
\usepackage{hyperref}
\hypersetup{colorlinks=true, linkcolor=blue, urlcolor=cyan}
\urlstyle{same}

\title{Artistic Research Milestone}
\author{Eirik Lie Hegre}

\begin{document}
\maketitle

\section{About My Sharing Format}
The format, for a lack of better word, I chose is a website called
\href{https://Github.com}{Github}. Github is website based on a software
called git, but it tries to simplify it somewhat by being a website while still
being fully compatible with the software. So what is git then? Git is a
\textit{version-control-software}, which is computer scientist for a
tool that helps people collaborate and being up to date with each other. With
git you can download a project, modify it, share it with the creator, who can
then choose to discard or include the changes. This means that it is really good
at including others in the project. If someone stops developing a project,
others can, to use the proper terminology, fork it, and take over the
development or take it in a new direction.
\par
When I first got into video design, I was drawn to certain software and effects
that was third party creations. Many of these had their own websites, but when
it came to download I had to go to Github, and this is where the confusion
began. If you are not familiar wit the git software, or probably even
simply enough not a computer scientist/programmer, a lot of the terminology and
user interfaces are really confusing. Even just figuring out how to download
and install something from Github took me quite some time. I have included a
small guide on my web page on how to download my project to alleviate this
problem.
\par
So having covered what it is, there is also a why. I want my project to be
accessible to others with an easy way of collaborating. I realize that my
interests are not necessarily the most common, so to increase the chance of
collaboration I turn to the world wide web. This way, anyone anywhere with an
internet connection. It also makes the project into a living entity, if my
interests change, I can store my previous work and keep it available to others
while changing the focus of my project. And all of this is for free. Another
big reason that I chose it for my work is that anytime I upload something I
have to write a message explaining what I have changed since the last time.
This means that it also functions as a automatic documenting tool as well
with anyone being able to see the changes as the project has developed.

\section{About My Project}
Now that I have covered what Github is, I will also describe my project. This
will be in brief, since I also cover it on my Github repository. My current
project is called \textit{Painting with Data}. I got the idea by experimenting
with a technique called databending. Databending is about opening images,
videos, and audio files in software they are not meant to be opened in, and then
changed. I have chosen to work with the jpeg file format, and to bend it by
opening it in a hex editor. This was to narrow the scope of the project down.
\par
Initially the plan was to find a way to databend files live as a way of creating
live visuals. So the start of the project also included to find software that
could help with the live performance as well. These experiments were not
fruitful and before I got really into it, the corona lockdown meant that my
possibilities of trying this live was cut woefully short. Instead the project
became more focused on the abilities and possibilities of the jpeg format with a
hope of finding more predictable ways of bending it.
\par
Another thing that has been apart of the project is the wish for it to be open
source and this is also why I chose to host my work on Github. There I also
licensed it under the GPL-3.0, a hard copyleft licence, so that it is available
for others to work on without possible legal ramifications. This is also
important since apart of what I am trying is to make people look at their
computer in a different way.
\par
The way I chose to do that was through bending the jpeg-format as previously
mentioned. What I am doing is using man readable text inside an image's source
data to create a new image that is a computer's interpretation of the man
readable text into something that makes sense for the machine. So much of
written down thoughts I have had are hidden inside of images for others to
explore.
\par
\medskip
\noindent But without much more ado, here is the link for my
\href{https://github.com/aa-83/artistic_research}{Artistic Research} for you
to explore.

\end{document}
