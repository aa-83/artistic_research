\documentclass[12pt, letterpaper]{article}
\usepackage[utf8]{inputenc}
\usepackage[norsk]{babel}
\usepackage[T1]{fontenc}

\usepackage[
backend=biber,
style=authoryear-icomp
]{biblatex}
\addbibresource{~/documents/latex/bib/bibliografi.bib}
\usepackage{csquotes}

\title{Findings}
\author{Eirik Lie Hegre}

\begin{document}
\maketitle

In this project I have some findigs that are expected, and some more unexpected.
One of the first things that I wanted to figure out was if any image would react
the same way to the same data input. This I was able to debunk early in the
process. In fact, the pattern folder clearly shows that the four different
images that I have been using got four different results even if they were based
on the same sonnet by Shakespeare. \par
Another thing I wanted to find out was if I could find specific bytes that would
make certain colours, or horizontal/vertical lines, or more pixelated blocks.
This was shown not to be the case in the experiments I made in the alphabet
folder. Even if the images only contained repetitions of one letter, the images
was anything but repetitive, but just as chaotic in expression as the other
images I made. \par
This made me re-prioritise some things in the project. I gave up any notion of
trying to gain any control of the process. Instead it would be more about bulk
creation and thorugh this be able to pick the images I like best. \par
One thing that would make this process easier would be to find a way to automate
this process. There is probably a script that I could write or a simple program,
but this will be sometime in the future.

\end{document}
