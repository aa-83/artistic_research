\documentclass[12pt, letterpaper]{article}
\usepackage[utf8]{inputenc}
\usepackage[english]{babel}
\usepackage[T1]{fontenc}
\usepackage{csquotes}
\usepackage{hyperref}
\hypersetup{colorlinks=true, linkcolor=blue, urlcolor=cyan}
\urlstyle{same}

\title{Artistic Research Milestone}
\author{Eirik Lie Hegre}

\begin{document}
\maketitle

\section{About My Sharing Format}
The format, for a lack of better word, I chose is a website called
\href{https://github.com}{Github}. Github is website based on a software
called git. So what is git then? Git is a \textit{version-control-software},
which is computer scientist for a cloud based sharing platform that help people
collaborate and be up to date with each other on a given project. With
git you can download a project, modify it, share it with the creator, who can
then choose to discard or include the changes. Another aspect of Github is that
others can pick up a project that has stopped being developed. To use the proper
terminology, \textit{fork it}, and take over the development.
\par
To contextualise my work with Github I will now briefly explain my own first
impression, and then why I chose to use this platform myself. When I first got
into video design, I was drawn to certain specialized software and effects
that was made for other existing software. These softwares were often hosted on
Github. If you are not familiar with the \textit{git} software,
 a lot of the terminology and
user interfaces are really confusing. Even just figuring out how to download
and install something from Github took me quite some time. I have included a
small guide on my web page on how to download my project to alleviate this
problem.
\par
So having covered what it is, there is also a why. I want my project to be
accessible to others with an easy way of collaborating. To increase the chance of
collaboration I turn to the world wide web. This way, anyone anywhere with an
internet connection can contribute to my project. It also turns the project into
a living entity. An interesting aspect of working this way is that if my
focus change, I can store my previous work and keep it available to others
while I am changing the main focus of my project. And all of this is for free.
Another big reason that I chose it for my work is that anytime I upload
something I have to write a message explaining what I have changed since the
last time. This means that it also functions as an automatic documenting tool
that makes anyone able to see the changes I have made as the project has
developed.

\section{About My Project}
My current project is called \textit{Painting with Data}. I got the idea by
experimenting
with a technique called databending. This is about opening images,
videos, and audio files in software they are not meant to be opened in, and then
change the files in these non---suited softwares. For example opening an image
in a Word document. I have chosen to work with the jpeg  which is an
image file format that uses a high compression algorithm to reduce its file
size. Then I bend these images by opening them in software type called
hex editor. This is to narrow the scope of the project down.
\par
Initially the plan was to find a method of bending files live as a way of
creating
live visuals. So the start of the project also included attempts to find
software that
could help with the live performance as well. These experiments were not
fruitful and before I got properly into it, the corona lockdown meant that my
possibilities of trying this live was cut woefully short. Instead the project
became more focused on the abilities and possibilities of the jpeg with a
hope of finding more predictable ways of bending it.
\par
Another thing that has been a part of the project is the wish for it to be open
source and this is also why I chose to host my work on Github. Where I have
licensed it under the GPL-3.0, a hard copyleft licence, so that it is available
for others to work on without possible legal ramifications.
\par
What I am trying is to make people look at their computer in a different way.
Unveiling the codes and algorithms that represent the machinery inside the
computer. The way I chose to do that is through bending the jpeg. I am using
human readable text inside an image's source data to create a new image. This
gives a computer's interpretation of the human readable text. Showing that for
the machine the content of the file is not important. So much of my written down
thoughts are hidden inside of images to illustrate this fact.
\par
\medskip
\noindent But without much more ado, here is the link for my
\href{https://github.com/aa-83/artistic_research}{Artistic Research} for you
to explore.

\end{document}
