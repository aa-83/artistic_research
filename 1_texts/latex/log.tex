\documentclass[12pt, letterpaper]{article}
\usepackage[utf8]{inputenc}
\usepackage[norsk]{babel}
\usepackage[T1]{fontenc}

\usepackage[
backend=biber,
style=authoryear-icomp
]{biblatex}
\addbibresource{~/documents/latex/bib/bibliografi.bib}
\usepackage{csquotes}

\title{Log}
\author{Eirik Lie Hegre}

\begin{document}
\maketitle

\tableofcontents

\section{2020--03--24}

\subsection{Brief summary}
My goal is to try and figure out more in depth about the jpg file format and how
I can databend it. Hopefully with some predictable results. If I open an image
and start randomly changing bytes the results often come out somehow similar.
Bright saturated colours, horizontal lines, compression artefacts, and
repetitions of the image content amongst other things.

\subsection{Databending}
The kind of databending I want to do is by opening the images in a hex editor
and edit the ASCII text. I want to see if there are ways that I can affect the
image in a predictable way by finding out more about how the format works. To do
that I will have to do some experiments.

\subsection{Method and Experiments}
My method will include finding four images, two that are different, but the same
resolution, and two more that are in different sizes. To sum up, four images,
three resolutions. These will all be scrubbed of data in a hex editor, down to
the size that removal of another byte will cause the file to break. By only
keeping the header of the original images I hope I will have a more common
ground to start from. The source for my images will be NASA since their images
are licensed under the Creative Commons.

So first of all I will scrub the data from the images. This turns the images
into grey canvases for the new data to be reapplied. Having done that I will
start writing in new data in the form of ASCII in a hex editor. With the text
that I write I will see if I can find patterns in how the computer interprets
the data I write into the image. An easy experiment will be to discover how many
characters make a line, I expect this to be dependent on resolution. After that
it will be to see if I can predict the colour as well. These things will be
tried out on all four images at the same time so that I will be able to
continuously compare the results of my changes.

\section{2020--03--25}
Made the four templates, copied them, and made test files for my experiments.
Having scrubbed the templates down to the point where a single more deleted byte
would break the file. This left me with four templates of the following sizes,
template1 is 1041$\times$694, template2 is 4053$\times$3461, template3 and template4 are both
1280$\times$720. These were all copied into four files, test1--4 for easy experimenting.
What I wanted to try first was to see how many characters do I need
make a line of pixels in the particular image I am editing. I expected the
different sized images to behave differently, but I found out the same---sized
also acted differently. Not only in colour, but what made one line test3, made
several lines in test4.

\section{2020--03--26}
Did not do much today, but added a license, GPL v.3.0 and added text to the
README

\section{2020--04--04}
Long Break in the work. Worked on an application for exchange in the fall, and
did not get back in the rythm afterwards. But I am back in business now! I need
to produce more content, and today I started writing about some of my thoughts
regarding open source. Each image adds a new line of pixels to the existing
image. In the end I hope to have a full image, and maybe make som animations
with them.

\section{2020--04--05}
I renamed notes.txt to log and moved it to the main directory. In the different
image directories I created new notes.txt files for each with the plan to write
my ideas for each specific folder. I also made 29 images in the Alphabet
directory.

\section{2020--04--06}
I have started updating the log and the notes.

\section{2020--04--07}
Made two new test directories. One named non\_repeating\_text and one
named metadata\_experiments. The plan for these directories would be to try out
metadata manipulation and just feeding random key strokes into an image. I
started out with the random key strokes and I ended up with a broken image. I
will try it one more time, but it seems like just random typing might break an
image.

\section{2020--04--08}
Today I worked with a program called exiv2 which is a metadata manipulation
program. I had hoped by using it I could make two seemingly similar images act
differently with the same sets of data put into them. That did not happen. Two
identical images will still look the same even if I manipulate the metadata in
one of them. I did not delve very deep into this though. So I might get some new
results with some further exploration.

\section{2020--04--09}
I did not do any work today, had the need for a think about what to do next.

\section{2020--04--10}
I wanted to readress the non---repeating text experiments, and turn them into the
direct opposite. I found out that it was time consuming putting in the
non---repeating text and have decided on using repeating patterns of text instead.
First out I will use a sonnet by Shakespeare and see where it will take me.

\section{2020--04--13}
Took a break over the weekend. Today I finished the sonnet experiments. But
instead of only applying one template to the first ten sonnets, I used all four
template instead.

\end{document}
