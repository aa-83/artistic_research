\documentclass[12pt, letterpaper]{article}
\usepackage[utf8]{inputenc}
\usepackage[norsk]{babel}
\usepackage[T1]{fontenc}

\usepackage[
backend=biber,
style=authoryear-icomp
]{biblatex}
\addbibresource{~/documents/latex/bib/bibliografi.bib}
\usepackage{csquotes}

\title{How to Read and View Images}
\author{Eirik Lie Hegre}

\begin{document}
\maketitle

\tableofcontents

\newpage

\section{How to Interact With Images}
There are to ways to open these images, as text files, and as an image. The
files I make will not behave like they are supposed to. This is because they are
all random assortments of data that are not supposed to be where they are,
according to the computer at least. This means that opening images in different
softwares will produce different results, despite the file being the same. This
is due to different software are based on different programming languages that
all interpret data in differently, but in they end up with the same result. But
because my images have no result, they are the results in themselves and are
trying to make one specific image. \par
If you want to read the text, depending on the software used, there will be
blocks of gibberish symbols being displayed. But somewhere inside the images
there will be the text I have written. Where this is though depends on the
software. \par
When it comes to view the image as an image, the most different results are to
be expected. I have noticed that opening the images on the website through a
browser generally produces the least interesting results. But downloading the
images and opening them in the operating systems standard image viewer creates
on average nice results. For each image viewer software used, a new result will
be produced, but in essence the image is the same since it is made up of the
same data.

\section{Reading the images}
\noindent To be able read the JPEGS they have to be opened in a text editor, hex
editor with ASCII support, or a word processor. Opening the images in these
different softwares I am listing will create widely different results. Some will
prompt you to open the images in certain file formats. I have not tested all of
these softwares myself, so I am not sure how all will act, but they should all
work. \bigskip

\subsection{Examples of hex editors}
\begin{itemize}
	\item Bless (Linux)
	\item Hexedit (MacOS)
	\item WinHex (Windows)
\end{itemize}

\subsection{Examples of text editors}
\begin{itemize}
	\item Vim
	\item Emacs
	\item Notepad (Windows)
	\item TextEdit (MacOS)
\end{itemize}

\subsection{Examples of word processors}
\begin{itemize}
	\item Microsoft Word
	\item OpenOffice Writer
	\item LibreOffice Writer
	\item Apple Pages (MacOS)
	\item Google Docs
\end{itemize}

\subsection{Notes}
Any of these programs should be able to open the images I have made as text
files. Inside the
images there is a lot of text. The reason is because if not, the images would
mostly be a grey solid. In images where I have repeated text, like the sonnets I
have had to separate between the repetitions by using these
signs /* */, with the text being in between these signs. Also since images are not
supposed to be opened in text programs, the formatting will be weird, and it is
often necessary to scroll horizontally for text editors and vertically for word
processors.

\section{Previewing images}
As I mentioned earlier, this is where the most different results will come. The
images in this folder were made on my computer with a hex editor called bless
and previewed with an image viewer called sxiv. Unless you use sxiv as well, the
image will not look the same for you as it did for me when I made it. I will
recommend
trying different softwares and see how the images change based on software.
Zooming in and out on the images can give interesting results as well depending
on the image viewer creates the zoom, if it scaling the image or through other
means.

\subsection{Examples of Windows Image Viewers}
\begin{itemize}
	\item Windows Photos
	\item IrfanView
	\item FastStone Image Viewer
	\item XnView MP
	\item HoneyView
	\item Nomacs
	\item Imagine Picture Viewer
	\item Gimp
	\item Photoshop
\end{itemize}

\subsection{Examples of Mac OSX Image Viewers}
\begin{itemize}
	\item Preview
	\item XnView MP
	\item ApolloOne
	\item qView
	\item Fragment
	\item Lyn
	\item Gimp
	\item Photoshop
	\item feg (CLI)
	\item sxiv (CLI)
\end{itemize}

\subsection{Examples of Linux Image Viewers}
\begin{itemize}
	\item Nomacs
	\item Gwenview
	\item Ristretto
	\item Geeqie
	\item Mirage
	\item Gimp
	\item feh (CLI)
	\item sxiv (CLI)
\end{itemize}




\end{document}

